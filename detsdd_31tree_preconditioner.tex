%%stopskip  : marker for tim's thesis, please do not remove it :-)


\subsection{A first preconditioner\label{sec:A-first-preconditioner}}

While the results in this section are not the main claims of this
paper, we hope they will provide some intuition, and an easier path
towards an implementation.

We present a first preconditioner that is not optimal, but that will
motivate our results for stronger preconditioners: a tree that spans
the graph $G$. Every graph has a low-stretch spanning tree, as discovered
by Alon et al. \cite{Alon1995}. The bound of Alon et al. was then
improved by Abraham et al. \cite{Abraham2008}. We restate their main
result.

\begin{lemma}(Lemma 9.2 from \cite{Spielman2009a}). Consider a weighted
graph $G$. There exists a spanning tree $T$ that is a subgraph of
$G$ so that: 
\[
L_{T}\preceq L_{G}\preceq\kappa L_{T}
\]
with $\kappa=\tilde{\mathcal{O}}\left(m\log n\right)$. %$\kappa=Cm\log n\log\log n\left(\log\log\log n\right)^{3}$ for some constant $C>0$.
\label{lem:tree-st} \end{lemma}

\begin{proof} 

% TODO: this proof uses stretch, and still referes to the old bound.

This follows directly from \cite{Spielman2009a}. $T$ is a subgraph
of $G$ (with the same weights on the edges), so $L_{T}\preceq L_{G}$
(see \cite{Spielman2009a} for example for a proof of this fact).
Furthermore, we have $L_{G}\preceq\text{st}_{T}\left(G\right)L_{T}$.
This latter inequality is a result of Spielman et al. in \cite{Spielman2010}
that we generalize in Lemma \ref{lem:stretch-inequality}. Finally,
a result by \cite{Abraham2008} shows that $T$ can be chosen such
that $\text{st}_{T}\left(G\right)\leq\mathcal{O}(m\log n(\log\log n)^{3})$.\end{proof}

Trees enjoy a lot of convenient properties for Gaussian elimination.
The Cholesky factorization of a tree can be computed in linear time,
and furthermore this factorization has a linear number of non-zero
elements \cite{Spielman2009a}. This factorization can be expressed
as: 
\[
L_{T}=PLDL^{T}P^{T}
\]
where $P$ is a permutation matrix, $L$ is a lower-triangular matrix
with the diagonal being all ones, and $D$ a diagonal matrix in which
all the elements but the last one are positive, the last element being
$0$. These well-known facts about trees are presented in \cite{Spielman2009a}.
Once the Cholesky factorization of the tree is performed, the log-determinant
of the original graph is an immediate by-product: 
\[
\log\left|L_{T}\right|=\sum_{i=1}^{n-1}\log D_{ii}
\]
Furthermore, computing $L_{T}^{+}x$ also takes $O\left(n\right)$
computations by forward-backward substitution (see \cite{duff1986direct}).
Combining Corollary \ref{cor:preconditioning} and Lemma \eqref{lem:tree-st}
gives immediately the following result.

\begin{theorem} \label{thm:PLD-tree}Let $G$ be a graph with $n$
vertices and $m$ edges. Its PLD can be computed up to a precision
$\epsilon$ and with high probability in time $\tilde{O}\left(m^{2}\log n\log^{2} \left(m\right) \left(\frac{1}{\epsilon}+\frac{1}{n\epsilon^{2}}\right) \log\left(\frac{2m}{\epsilon}\right) \log\left(2/\eta\right)\right)$.\end{theorem}

\begin{proof} Using Lemma \eqref{lem:tree-st}, we compute a low-stretch
tree $L_{T}$ so that $L_{T}\preceq L_{G}\preceq\kappa L_{T}$ with
$\kappa=\tilde{\mathcal{O}}\left(m\log n\right)$. Using Corollary
\eqref{cor:preconditioning}, approximating the PLD with high precision
requires 
\begin{eqnarray*}
\tilde{O}\left(\kappa\left(\frac{1}{\epsilon}+\frac{1}{n\epsilon^{2}}\right)\log\left(\frac{2\kappa}{\epsilon}\right)\log\left(2/\eta\right)\log^{2}\left(\kappa\right)\right)\\
=\tilde{O}\left(m\log n\left(\frac{1}{\epsilon}+\frac{1}{n\epsilon^{2}}\right)\log\left(\frac{2m}{\epsilon}\right)\log\left(2/\eta\right)\log^{2}\left(m\right)\right)
\end{eqnarray*}
 inversions by the tree $T$ (done in $O\left(n\right)$) and vector
products by the floated Laplacian $F_{L_{G}}$(done in $O\left(m\right)$).
The overall cost is $\tilde{O}\left(m^{2}\log n\log^{2}\left(m\right)\left(\frac{1}{\epsilon}+\frac{1}{n\epsilon^{2}}\right)\log\left(\frac{2m}{\epsilon}\right)\log\left(2/\eta\right)\right)$.
\end{proof}

The previous result shows that the log-determinant can be compute
in roughly $\mathcal{O}\left(m^{2}\right)$ ($m$ being the number
of non-zero entries). This result may be of independent interest since
it requires relatively little machinery to compute, and it is a theoretical
improvement already for graphs with small vertex degree ($m=\mathcal{O}\left(n^{1+o\left(1\right)}\right)$)
over the Cholesky factorization of $G$ (which has complexity $\mathcal{O}\left(n^{3}\right)$
in all generality). Also, note that the PLD of the tree constructed
above provides an upper bound to the log-determinant of $G$ since
$L_{G}\preceq\kappa L_{T}$. We will see in Subsection \ref{sub:Stretch-bounds}
that we can compute a non-trivial lower bound as well. 