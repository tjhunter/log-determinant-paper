
\subsection{A first preconditioner\label{sec:A-first-preconditioner}}

While the results in this section are not the main claims of this
paper, we hope they will provide some intuition, and an easier path
towards an implementation.

We present a first preconditioner that is not optimal, but that will
motivate our results for stronger preconditioners: a tree that spans
the graph $G$. Every graph has a low-stretch spanning tree, as discovered
by Alon et al. \cite{Alon1995}. The bound of Alon et al. was then
improved by Abraham et al. \cite{Abraham2008}. We restate their main
result. \begin{lemma}(Theorem 1 from \cite{Abraham2008}). Consider
$L_{G}$ a weighted graph. There exists a tree $L_{T}$ that is a
subgraph of $L_{G}$ so that: 
\[
L_{G}\preceq L_{T}\preceq\kappa L_{G}
\]
with $\kappa = \tilde{\mathcal{O}}\left(m\log n\right)$. %$\kappa=Cm\log n\log\log n\left(\log\log\log n\right)^{3}$ for some constant $C>0$.
\end{lemma} 

\begin{proof} This follows directly from \cite{Abraham2008}. $L_{T}$
is a subgraph of $L_{G}$ (with the same weights on the edges), so
$L_{T}\preceq L_{G}$ (see \cite{Spielman2009a} for example for a
proof of this fact). Furthermore, we have $L_{T}\preceq\text{st}_{G}\left(T\right)L_{G}$.
This latter inequality is result of Boman and Hendrickson \cite{Boman2004}
cited by Spielman et al. in \cite{Spielman2010} that we generalize
in Lemma \ref{lem:stretch-inequality}. \end{proof}

Trees enjoy a lot of convenient properties for Gaussian elimination.
The Cholesky factorization of a tree can be computed in linear time,
and furthermore this factorization has a linear number of non-zero
elements \cite{Spielman2009a}. This factorization can be expressed
as: 
\[
L_{T}=PLDL^{T}P^{T}
\]
where $P$ is a permutation matrix, $L$ is a lower-triangular matrix
with the diagonal being all ones, and $D$ a diagonal matrix in which
all the elements but the last one are positive, the last element being
$0$. These well-known facts about trees are presented in \cite{Spielman2009a}.
Once the Cholesky factorization of the tree is performed, the log-determinant
of the original graph is an immediate by-product: 
\[
\log\left|L_{T}\right|=\sum_{i=1}^{n-1}\log D_{ii}
\]
Furthermore, computing $L_{T}^{+}x$ also takes $O\left(n\right)$
computations by forward-backward substitution. Applying Corollary
\ref{cor:preconditioning} gives immediately: 

\begin{theorem} \label{thm:PLD-tree}Let $G$ be a graph with $n$
vertices and $m$ edges. Its PLD can be computed up to a precision $\epsilon$ and with high probability in time $\tilde{O}\left(mn\left(\frac{1}{\epsilon}+\frac{1}{n\epsilon^{2}}\right)\left(\log m\right)^{3}\left(\log n\right)^{2}\log\epsilon^{-1}\right)$.\end{theorem}

\begin{proof} We assume $G$ is connected, hence $m\geq n$ and $\log\kappa=\tilde{\mathcal{O}}\left(m\log n\right)$.
\end{proof}

This result may be of independent interest since it requires relatively
little machinery to compute, and it is an improvement already for
graphs with small vertex degree ($m=\mathcal{O}\left(n^{1+o\left(1\right)}\right)$)
over the Cholesky factorization of $G$. Also, note that the PLD of
the tree constructed above provides an upper bound to the log-determinant
of $G$ since $L_{G}\preceq L_{T}$. We will see in Subsection \ref{sub:Stretch-bounds}
that we can compute a non-trivial lower bound as well.
