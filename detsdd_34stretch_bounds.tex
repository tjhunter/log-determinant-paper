
\subsection{Stretch bounds on preconditioners\label{sub:Stretch-bounds}}

How good is the estimate provided by the preconditioner? Intuitively,
this depends on how well the preconditioner $L_{H}$ approximates
the graph $L_{G}$. This notion of quality of approximation can be
formalized by the notion of \emph{stretch}. This section presents
a deterministic bound on the PLD of $L_{G}$ based on the PLD of $L_{H}$
and the stretch of $G$ relative to $H$. This may be useful in practice
as it gives a (tight) interval for the PLD before performing any Monte-Carlo
estimation of the residual.

The stretch of a graph is usually defined with respect to a (spanning)
tree. In our analysis, it is convenient and straightforward to generalize
this definition to arbitrary graphs. To our knowledge, this straightforward
extension is not considered in the litterature, so we feel compelled
to properly introduce it.

\begin{definition}\emph{Generalized stretch}.\label{Generalized-stretch}
Consider $\mathcal{V}$ a set of vertices, $G=\left(\mathcal{V},\,\mathcal{E}_{G}\right),\, H=\left(\mathcal{V},\,\mathcal{E}_{H}\right)$
connected graphs over the same set of vertices, and $L_{G}$, $L_{H}$
their respective Laplacians. The stretch of $G$ with respect to $H$
is the sum of the effective resistances of each edge of graph $G$
with repect to graph $H$, 
\[
\text{st}_{H}\left(G\right)=\sum_{\left(u,v\right)\in\mathcal{E}_{G}}L_{G}\left(u,v\right)\left(\mathcal{X}_{u}-\mathcal{X}_{v}\right)^{T}L_{H}^{+}\left(\mathcal{X}_{u}-\mathcal{X}_{v}\right)
\]
with $\mathcal{X}_{u}\in\mathbb{R}^{n}$ the unit vector that is $1$
at position $u$, and zero otherwise.

\end{definition}

If the graph $H$ is a tree, this is a standard definition of stretch,
because the effective resistance $\left(\mathcal{X}_{u}-\mathcal{X}_{v}\right)^{T}L_{H}^{+}\left(\mathcal{X}_{u}-\mathcal{X}_{v}\right)$
between vertices $u$ and $v$ is the sum of all resistances over
the unique path between $u$ and $v$ (see Lemma 2.4 in \cite{Spielman2009b}).
Furthermore, the arguments to prove Theorem 2.1 in \cite{Spielman2009b}
carry over to our definition of stretch. For the sake of completeness,
we include this result:

\begin{lemma}\label{lem:stretch-trace}(Straightforward generalization
of Theorem 2.1 in \cite{Spielman2009b}) Let $G=\left(\mathcal{V},\,\mathcal{E}_{G}\right),\, H=\left(\mathcal{V},\,\mathcal{E}_{H}\right)$
be connected graphs over the same set of vertices, and $L_{G}$, $L_{H}$
their respective Laplacians. Then:
\[
\text{st}_{H}\left(G\right)=\text{Tr}\left(L_{H}^{+}L_{G}\right)
\]
with $L_{H}^{+}$the pseudo-inverse of $L_{H}$.

\end{lemma}

\begin{proof}

We denote $E\left(u,v\right)$ the Laplacian unit matrix that is $1$
in position $u,v$: $E\left(u,v\right)=\left(\mathcal{X}_{u}-\mathcal{X}_{v}\right)\left(\mathcal{X}_{u}-\mathcal{X}_{v}\right)^{T}$.
This is the same arguments as the original proof:
\begin{eqnarray*}
\text{Tr}\left(L_{H}^{+}L_{G}\right) & = & \sum_{\left(u,v\right)\in\mathcal{E}_{G}}L_{G}\left(u,v\right)\text{Tr}\left(E\left(u,v\right)L_{H}^{+}\right)\\
 & = & \sum_{\left(u,v\right)\in\mathcal{E}_{G}}L_{G}\left(u,v\right)\text{Tr}\left(\left(\mathcal{X}_{u}-\mathcal{X}_{v}\right)\left(\mathcal{X}_{u}-\mathcal{X}_{v}\right)^{T}L_{H}^{+}\right)\\
 & = & \sum_{\left(u,v\right)\in\mathcal{E}_{G}}L_{G}\left(u,v\right)\left(\mathcal{X}_{u}-\mathcal{X}_{v}\right)^{T}L_{H}^{+}\left(\mathcal{X}_{u}-\mathcal{X}_{v}\right)\\
 & = & \text{st}_{H}\left(G\right)
\end{eqnarray*}


\end{proof}

A consequence is $\text{st}_{H}\left(G\right)\geq\text{Card}\left(\mathcal{E}_{G}\right)\geq n-1$
for connected $G$ and $H$ with $L_{G}\succeq L_{H}$, and that for
any connected graph $G$, $\text{st}_{G}\left(G\right)=n-1$. Scaling
and matrix inequalities carry over with the stretch as well. Given
$A,B,C$ connected graphs, and $\alpha,\beta>0$:
\[
\text{st}_{\alpha A}\left(\beta B\right)=\alpha^{-1}\beta\text{st}_{A}\left(B\right)
\]


\[
L_{A}\preceq L_{B}\Rightarrow\text{st}_{A}\left(C\right)\geq\text{st}_{B}\left(C\right)
\]
\[
L_{A}\preceq L_{B}\Rightarrow\text{st}_{C}\left(A\right)\leq\text{st}_{C}\left(B\right)
\]


\begin{lemma}For any connected graph $G$, $\text{st}_{G}\left(G\right)=n-1$.

\end{lemma}

\begin{proof}Consider the diagonalization of $L_{G}$: $L_{G}=P\Delta P^{T}$
with $P\in\mathbb{R}^{n\times n-1}$ and $\Delta=\text{diag}\left(\lambda_{1}\cdots\lambda_{n-1}\right)$.
Then 
\[
\text{st}_{G}\left(G\right)=\text{Tr}\left(P\Delta P^{T}P\Delta^{-1}P^{T}\right)=\text{Tr}\left(\Delta\Delta^{-1}\right)=n-1
\]


\end{proof}

A number of properties of the stretch extend to general graphs using
the generalized stretch. In particular, the stretch inequality (Lemma
8.2 in \cite{Spielman2009a}) can be generalized to arbitrary graphs
(instead of spanning trees).

\begin{lemma}\label{lem:stretch-inequality}Let $G=\left(\mathcal{V},\,\mathcal{E}_{G}\right),\, H=\left(\mathcal{V},\,\mathcal{E}_{H}\right)$
be connected graphs over the same set of vertices, and $L_{G}$, $L_{H}$
their respective Laplacians. Then:
\[
L_{G}\preceq\text{st}_{H}\left(G\right)L_{H}
\]


\end{lemma}

\begin{proof}The proof is very similar to that of Lemma 8.2 in \cite{Spielman2009b},
except that the invocation of Lemma 8.1 is replaced by invoking Lemma
\ref{lem:simple-inequality} in Appendix B. The Laplacian $G$ can
be written as a linear combination of edge Laplacian matrices:
\[
L_{G}=\sum_{e\in\mathcal{E}_{G}}\omega_{e}L\left(e\right)=\sum_{\left(u,v\right)\in\mathcal{E}_{G}}\omega_{\left(u,v\right)}\left(\mathcal{X}_{u}-\mathcal{X}_{v}\right)\left(\mathcal{X}_{u}-\mathcal{X}_{v}\right)^{T}
\]
and using Lemma \ref{lem:simple-inequality} gives:
\[
\left(\mathcal{X}_{u}-\mathcal{X}_{v}\right)\left(\mathcal{X}_{u}-\mathcal{X}_{v}\right)^{T}\preceq\left(\mathcal{X}_{u}-\mathcal{X}_{v}\right)^{T}L_{H}^{+}\left(\mathcal{X}_{u}-\mathcal{X}_{v}\right)L_{H}
\]
By summing all the edge inequalities, we get:
\begin{eqnarray*}
L_{G} & \preceq & \sum_{\left(u,v\right)\in\mathcal{E}_{G}}\omega_{\left(u,v\right)}\left(\mathcal{X}_{u}-\mathcal{X}_{v}\right)^{T}L_{H}^{+}\left(\mathcal{X}_{u}-\mathcal{X}_{v}\right)L_{H}\\
 & \preceq & \text{st}_{H}\left(G\right)L_{H}
\end{eqnarray*}


\end{proof}

This bound is remarkable as it relates any pair of (connected) graphs,
as opposed to spanning trees or subgraphs. An approximation of the
generalized stretch can be quickly computed using a construct detailled
in \cite{Spielman2009}, as we will see below. We now introduce the
main result of this section: a bound on the PLD of $L_{G}$ using
the PLD of $L_{H}$ and the stretch.

\begin{theorem}\label{thm:stretch-pld-bounds}Let $G=\left(\mathcal{V},\,\mathcal{E}_{G}\right),\, H=\left(\mathcal{V},\,\mathcal{E}_{H}\right)$
be connected graphs over the same set of vertices, and $L_{G}$, $L_{H}$
their respective Laplacians. Assuming $L_{H}\preceq L_{G}$, then:
\begin{equation}
\text{ld}\left(L_{H}\right)+\left(n-1\right)\log\left(\frac{\text{st}_{H}\left(G\right)}{n-1}\right)\geq\text{ld}\left(L_{G}\right)\geq\text{ld}\left(L_{H}\right)+\log\left(\text{st}_{H}\left(G\right)-n+2\right)\label{eq:encadrement-1}
\end{equation}


This bound is tight.

\end{theorem}

\begin{proof} This is an application of Jensen's inequality on $\text{ld}\left(L_{H}^{+}L_{G}\right)$.
We have $\text{ld}L_{G}=\text{ld}L_{H}+\text{ld}\left(L_{H}^{+}G\right)$
and $\text{ld}\left(L_{H}^{+}G\right)=\text{ld}\left(\sqrt{L_{H}}^{+}L_{G}\sqrt{L_{H}}^{+}\right)$
with $\sqrt{T}$ the matrix square root of $T$. From Lemma \ref{lem:Jensen-inequality-matrix-logarithm-1},
we have the following inequality: 
\begin{eqnarray*}
\text{ld}\left(\sqrt{L_{H}}^{+}L_{G}\sqrt{L_{H}}^{+}\right) & \leq & \left(n-1\right)\log\left(\frac{\text{Tr}\left(\sqrt{L_{H}}^{+}L_{G}\sqrt{L_{H}}^{+}\right)}{n-1}\right)\\
 & = & \left(n-1\right)\log\left(\frac{\text{Tr}\left(L_{H}^{+}L_{G}\right)}{n-1}\right)\\
 & = & \left(n-1\right)\log\left(\frac{\text{st}_{H}\left(G\right)}{n-1}\right)
\end{eqnarray*}


The latter equality is an application of Lemma \ref{lem:stretch-trace}. 

The lower bound is slightly more involved. Call $\lambda_{i}$ the
positive eigenvalues of $\sqrt{L_{H}}^{+}L_{G}\sqrt{L_{H}}^{+}$ and
$\sigma=\text{st}_{H}\left(G\right)$. We have $1\leq\lambda_{i}$
from the asumption $L_{H}\preceq L_{G}$. By definition: $\text{ld}\left(L_{H}^{+}L_{G}\right)=\sum_{i}\log\lambda_{i}$.
Furthermore, we know from Lemma \ref{lem:stretch-trace} that $\sum_{i}\lambda_{i}=\sigma$.
The upper and lower bounds on $\lambda_{i}$ give:
\[
\begin{array}{cc}
\text{ld}\left(L_{H}^{+}L_{G}\right)\geq & \min\,\sum_{i}\log\lambda_{i}\\
 & 1\leq\lambda_{i}\\
 & \sum_{i}\lambda_{i}=\sigma
\end{array}
\]
Since there are precisely $n-1$ positive eigenvalues $\lambda_{i}$,
one can show that the minimization problem above has a unique minimum
which is $\log\left(\sigma-n+2\right)$.

To see that, consider the equivalent problem of minimizing $\sum_{i}\log\left(1+u_{i}\right)$
under the constraints $\sum_{i}u_{i}=\sigma-\left(n-1\right)$ and
$u_{i}\geq0$. Note that:
\[
\sum_{i}\log\left(1+u_{i}\right)=\log\left(\prod_{i}\left[1+u_{i}\right]\right)=\log\left(1+\sum_{i}u_{i}+\text{Poly}\left(u\right)\right)
\]
with $\text{Poly}\left(u\right)\geq0$ for all $u_{i}\geq0$, so we
get: $\sum_{i}\log\left(1+u_{i}\right)\geq\log\left(1+\sum_{i}u_{i}\right)$
and this inequality is tight for $u_{1}=\sigma-\left(n-1\right)$
and $u_{i\geq2}=0$. Thus the vector $\lambda^{*}=\left(\sigma-n+2,\,1\cdots1\right)^{T}$
is (a) a solution to the minimization problem above, and (b) the objective
value of any feasible vector $\lambda$ is higher or equal. Thus,
this is the solution (unique up to a permutation). Hence we have $\text{ld}\left(L_{H}^{+}L_{G}\right)\geq\sum_{i}\log\lambda_{i}^{*}=\log\left(\sigma-n+2\right)$.

Finally, note that if $H=G$, then $\text{st}_{H}\left(G\right)=n-1$,
which gives an equality.\end{proof}

Note that Lemma \ref{lem:stretch-inequality} gives us $L_{H}\preceq L_{G}\preceq\text{st}_{H}\left(G\right)L_{H}$
which implies $\text{ld}\left(L_{H}\right)\leq\text{ld}\left(L_{G}\right)\leq\text{ld}\left(L_{H}\right)+n\log\text{st}_{H}\left(G\right)$.
The inequalities in Theorem \ref{thm:stretch-pld-bounds} are stronger.
Interestingly, it does not make assumption on the topology of the
graphs (such as $L_{H}$ being a subset of $L_{G}$). Research on
conditioners has focused so far on low-stretch approximations that
are subgraphs of the original graph. It remains to be seen if some
better preconditioners can be found with stretches in $\mathcal{O}\left(n\right)$
by considering more general graphs. In this case, the machinery developped
in Section 3 would not be necessary.

From a practical perspective, the stretch can be calculated also with
respect to the number of non-zero entries.

\begin{lemma}\label{lem:stretch-approx}Let $G=\left(\mathcal{V},\,\mathcal{E}_{G}\right),\, H=\left(\mathcal{V},\,\mathcal{E}_{H}\right)$
be connected graphs over the same set of vertices, and $L_{G}$, $L_{H}$
their respective Laplacians. Call $r=\max_{e}L_{H}\left(e\right)/\min_{e}L_{H}\left(e\right)$.
Given $\epsilon>0$, there exists an algorithm that returns a scalar
$y$ so that:
\[
\left(1-\epsilon\right)\text{st}_{H}\left(G\right)\leq y\leq\left(1+\epsilon\right)\text{st}_{H}\left(G\right)
\]
with high probability and in expected time $\tilde{\mathcal{O}}\left(m\epsilon^{-2}\log\left(rn\right)\right)$.

\end{lemma}

\begin{proof}This is a straightforward consequence of Theorem $2$
in \cite{Spielman2009}. Once the effective resistance of an edge
can be approximated in time $O\left(\log n/\epsilon^{2}\right)$,
we can sum it and weight it by the conductance in $G$ for each edge.

\end{proof}


\subsection{Fast inexact estimates}

The bound presented in Equation \ref{eq:encadrement-1} has some interesting
consequences if one is interested only in a rough estimate of the
log-determinant: if $\epsilon=O\left(1\right)$, it is possible to
approximate the log-determinant in expected time $\tilde{O}\left(m+n\log^{3}n\right)$.
We will make use of this sparsification result from Spielman and Srivastava:

\begin{lemma}\label{lem:sriva-sparsification}(Theorem 12 in \cite{Spielman2009}).
Given a Laplacian $L_{G}$ with $m$ edges, there is an expected $\tilde{O}\left(m/\epsilon^{2}\right)$
algorithm that produces a graph $L_{H}$ with $O\left(n\log n/\epsilon^{2}\right)$
edges that satisfies $\left(1-\epsilon\right)L_{G}\preceq L_{H}\preceq\left(1+\epsilon\right)L_{G}$.

\end{lemma}

An immediate consequence is that given any graph, we can find a graph
with a near-optimal stretch (up to an $\epsilon$ factor) and $O\left(n\log n/\epsilon^{2}\right)$
edges.

\begin{lemma}\label{lem:low-stretch-bounding}Given a Laplacian $L_{G}$
with $m$ edges, there is an expected $\tilde{O}\left(m/\epsilon^{2}\right)$
algorithm that produces a graph $L_{H}$ with $O\left(n\log n/\epsilon^{2}\right)$
edges that satisfies $\left(n-1\right)\leq\text{st}_{H}\left(G\right)\preceq\frac{1+\epsilon}{1-\epsilon}\left(n-1\right)$.

\end{lemma}

\begin{proof}Consider a graph $H$ produced by Lemma \ref{lem:sriva-sparsification},
which verifies $\left(1-\epsilon\right)L_{G}\preceq L_{H}\preceq\left(1+\epsilon\right)L_{G}$.
Using the stretch over this matrix inequality, this implies:
\[
\text{st}_{\left(1-\epsilon\right)G}\left(G\right)\geq\text{st}_{H}\left(G\right)\geq\text{st}_{\left(1+\epsilon\right)G}\left(G\right)
\]
which is equivalent to:
\[
\left(1-\epsilon\right)^{-1}\text{st}_{G}\left(G\right)\geq\text{st}_{H}\left(G\right)\geq\left(1+\epsilon\right)^{-1}\text{st}_{G}\left(G\right)
\]
and the stretch of a connected graph with respect to itself is $n-1$.
By rescaling $H$ to $\left(1+\epsilon\right)^{-1}H$, we get:
\[
\frac{1+\epsilon}{1-\epsilon}\left(n-1\right)\geq\text{st}_{H}\left(G\right)\geq n-1
\]


\end{proof}

Here is the main result of this section:

\begin{proposition}There exists an algorithm that on input $A\in SDD_{n}$,
returns an approximation $n^{-1}\log\left|A\right|$ with precision
$1/2$ in expected time $\tilde{O}\left(m+n\log^{3}n\log^{2}\kappa_{A}\right)$
with $\kappa_{A}$ the condition number of $A$.

\end{proposition}

\begin{proof}Given $L_{A}$, compute $H$ from Lemma \ref{lem:low-stretch-bounding}
using $\epsilon=1/16$ so that $\left(n-1\right)\leq\text{st}_{H}\left(G\right)\preceq\left(1+1/8\right)\left(n-1\right)$.
Then, using Theorem \ref{thm:stretch-pld-bounds}, this leads to the
bound:
\[
\text{\text{ld}}\left(H\right)\leq\log\left|A\right|\leq\text{ld}\left(H\right)+\frac{n-1}{4}
\]
since $H$ has $O\left(n\log n\right)$ edges by construction, we
can use Theorem \ref{thm:ultra_main} to compute a $1/4$- approximation
of $\text{ld}\left(H\right)$ in expected time $\tilde{O}\left(n\log^{3}n\log^{2}\left(\kappa_{H}\right)\right)$.
By construction $\kappa_{H}\leq\frac{1+1/16}{1-1/16}\kappa_{A}$,
hence the result.

\end{proof}

It would be interesting to see if this technique could be developed
to handle arbitrary precision as well.
