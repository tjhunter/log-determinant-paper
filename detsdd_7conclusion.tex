
\section*{Comments}

A first approximation upper-bound of the log-determinant follows immediately
from the computation of the $\kappa-$good chain. We presented in
Section \ref{sec:A-first-preconditioner} a first analysis to bound
the value of the residue PLD. This analysis was done using trees as
preconditioners, it can be carried on on more general preconditioners
by introducing a generalization of the stretch over a subgraph (see
the Appendix, section \ref{sub:Stretch-graph}). Since the bulk of
the computations are performed in estimating the residue PLD, it would
be interesting to see if this could be bypassed using better bounds
based on the stretch.

When looking at each step of the analysis, one can see that the $\epsilon^{-2}$
factor comes from the approximation of the trace of a matrix by sampling.
This seems to be a fundamental limitation of this method and it is
shared with other algorithms that rely on random projections. This
result is absolute. It would be interesting to see if the analysis
could be tightened to present a relative bound that does not depend
on the condition number of $A$. Also, even if this algorithm presents
a linear bound, it requires a fairly advanced machinery (ST solvers)
that may limit its practicality. Some heuristic implementation, for
example based on algebraic multi-grid methods, could be a first step
in this direction.

The authors are much indebted to Satish Rao and Jim Demmel for suggesting
the original idea and their helpful comments on the draft of this
article.
