%%stopskip  : marker for tim's thesis, please do not remove it :-)



\subsection{Incremental sparsifiers\label{sec:Incremental-sparsifiers}}

We can do better and achieve near-linear time by using ultra-sparsifiers.
The main insight of our result is that the class preconditioners presented
by Spielman and Teng are based on incomplete Cholesky factorization,
and hence have a determinant that is relatively easy to compute, and
furthermore that they are excellent spectral preconditioners, so the
procedure \texttt{PreconditionedLogDet-MonteCarlo} is efficient to
apply. We reintroduce some concepts presented in \cite{Koutis2010}
to present a self-contained result. The following paragraphs are well-known
facts about Spielman-Teng preconditioners and have been presented
in \cite{Koutis2010,Spielman2009a}.

The central idea to the Spielman-Teng preconditioner is to sample
$\mathcal{O}\left(n\right)$ edges from the graph $A$, to form a
subgraph $B$ that is close to a tree (hence it is easy to compute
some partial Cholesky factorization), yet it is close to the original
$A$ is the spectral sense ($A\preceq B\preceq\kappa A$), thanks
to the additional edges. The partial Cholesky factorization is computed
using the \texttt{GreedyElimination} algorithm presented in \cite{Koutis2010}.
In order for this section to be self-contained, we include here the
main results of Section 4 in \cite{Spielman2009a}.

Consider the Laplacian matrix $L_{B}$ of the subgraph $B$. There
exists an algorithm that computes the partial Cholesky factorization:
\[
L_{B}=PLCL^{T}P^{T}
\]
where: 
\begin{itemize}
\item $P$ is a permutation matrix 
\item $L$ is a non-singular, low triangular matrix of the form 
\[
L=\left(\begin{array}{cc}
L_{1,1} & 0\\
L_{2,1} & I_{n_{1}}
\end{array}\right)
\]
with the diagonal of $L_{1,1}$ being all ones. 
\item $C$ has the form 
\[
C=\left(\begin{array}{cc}
D_{n-n_{1}} & 0\\
0 & L_{A_{1}}
\end{array}\right)
\]
and every row and column of $L_{A_{1}}$ has at least 3 non-zero coefficients.
Furthermore, $L_{A_{1}}$ is itself Laplacian and: 
\[
\text{ld}\left(L_{G}\right)=\sum_{1}^{n-n_{1}}\log D_{ii}+\mbox{\text{ld}}\left(L_{A_{1}}\right)
\]

\end{itemize}
The exact algorithm that achieves this factorization is called \texttt{GreedyElimination}
and is presented in \cite{Koutis2010}. Using this factorization,
the PLD of the original Laplacian $L_{A}$ is: 
\begin{eqnarray}
\text{ld}\left(L_{A}\right) & = & \text{ld}\left(L_{B}\right)+\text{ld}\left(B^{+}A\right)\nonumber \\
 & = & \sum_{1}^{n-n_{1}}\log D_{ii}+\mbox{\text{ld}}\left(A_{1}\right)+\text{ld}\left(B^{+}A\right)\label{eq:chain-recursion}
\end{eqnarray}
Thus, we are left with solving a smaller problem $A_{1}$, and we
approximate the value of $\text{ld}\left(B^{+}A\right)$ using the
algorithm \texttt{SampleLogDet}. ST preconditioners are appealing
for this task: they guarantee that $A_{1}$ is substantially smaller
than $A$, so the recursion completes in $\mathcal{O}\left(\log\log n\right)$
steps. Furthermore, computing the vector product $B^{+}Ax$ is itself
efficient (in can be done approximated in near-linear time), so we
can apply Theorem \ref{thm:preconditioning-approx}. We formalize
the notion of chain of preconditioners by reintroducing some material
from \cite{Koutis2010}.

\begin{algorithm}
\begin{raggedright} Algorithm \textbf{UltraLogDet}($A$,$\epsilon$,$\eta$):

\end{raggedright}

\begin{raggedright} If $A$ is of a small size ($<$100), directly
compute $\text{ld}\left(A\right)$ with a dense Cholesky factorization.

\end{raggedright}

\begin{raggedright} Compute $B=$\textbf{IncrementalSparsify($A$)}

\end{raggedright}

\begin{raggedright} Compute $D,A'=$\textbf{PartialCholesky($B$)}

\end{raggedright}% TODO: change the bounds to reflect the theorem


\begin{raggedright} $\eta\leftarrow\min\left(\frac{\epsilon}{8\kappa^{3}\kappa\left(B\right)},\frac{1}{2\kappa}\right)$

\end{raggedright}

\begin{raggedright} $p\leftarrow8\left(\frac{1}{\epsilon}+\frac{1}{n\epsilon^{2}}\right)\log\left(\eta^{-1}\right)\log^{2}\left(\delta^{-1}\right)$

\end{raggedright}

\begin{raggedright} $l\leftarrow\delta^{-1}\log\left(\frac{2}{\epsilon\delta}\right)$

\end{raggedright}

\begin{raggedright} Compute $s=$\textbf{PreconditionedLogDetMonteCarlo($B,A,\eta,p,l$)}

\end{raggedright}

\begin{raggedright} Return $s+\log\left|D\right|+$\textbf{UltraLogDet($A'$,$\epsilon$,$\eta$)}

\end{raggedright}

\caption{Sketch of the main algorithm\label{alg:The-main-algorithm}}
\end{algorithm}

\begin{definition}
Definition 4.2 from \cite{Koutis2011}. Good preconditioning chain.
Let $d\in\mathbb{N}^{*}$, $\mathcal{C}=\left\{ A_{1}=A,B_{1},A_{2},B_{2},A_{3}\dots B_{d-1},A_{d}\right\} $
be a chain of graphs and $\mathcal{K}=\left(\kappa_{1}\cdots\kappa_{d-1}\right)\in\mathbb{R}_{+}^{d-1}$.
We say that $\left\{ \mathcal{C},\mathcal{K}\right\} $ is a good
preconditioning chain for $A$ if there exists $\mathcal{U}=\left(\mu_{1}\cdots\mu_{d}\right)\in\mathbb{N}_{+}^{d}$
so that: 
\begin{enumerate}
\item $A_{i}\preceq B_{i}\preceq\kappa_{i}A_{i}$ . 
\item $A_{i+1}$= \texttt{GreedyElimination}$\left(B_{i}\right)$ . % TODO: We need to explain greedy elimination somewhere

\item The number of edges of $A_{i}$ is less than $\mu_{i}$. 
\item $\mu_{1}=\mu_{2}=m$ where $m$ is the number of edges of $A$. 
\item $\mu_{i}/\mu_{i+1}\geq c_{r}\left\lceil \sqrt{\kappa_{i}}\right\rceil $
for some constant $c_{r}$. 
\item $\kappa_{i+1}\leq\kappa_{i}$. 
\item $\mu_{d}$ is smaller than some fixed arbitrary constant. 
\end{enumerate}
\end{definition}
Good chains exist, as found by Koutis, Miller and Peng:
\begin{lemma}
\label{lem:good-chain} (Lemma 4.5 from \cite{Koutis2011}) Given
a graph $A$, the algorithm \texttt{BuildChain}$\left(A,p\right)$
from \cite{Koutis2011} produces with probability $1-p$ a good preconditioning
chain $\left\{ \mathcal{C},\mathcal{K}\right\} $ such that $\kappa_{1}=\tilde{O}\left(\log^{2}n\right)$
and $\kappa_{i}=\kappa_{c}$ for all $i\geq2$ for some constant $\kappa_{c}$.
The length of the chain is $d=\mathcal{O}\left(\log n\right)$ and
the algorithm runs in expected time $\tilde{O}\left(m\log n\right).$%+n\log n\log\log n\right)$

\end{lemma}
These chains furthermore can be used as good preconditioners for conjugate
gradient and lead to near-linear algorithms for approximate inversion
(Lemma 7.2 from \cite{Koutis2010}). This remarkable result has been
significantly strenghtened in the previous years, so that SDD systems
can be considered to be solved in (expected) linear time.
\begin{lemma}
\label{lem:linear-precond-existence}(Theorem 4.6 from \cite{Koutis2011}).
Given $A\in SDD_{n}$ with $m$ non-zero entries, $b\in\mathbb{R}^{n}$
and $\nu>0$, a vector $x$ such that $\left\Vert x-A^{+}b\right\Vert _{A}<\nu\left\Vert A^{+}b\right\Vert _{A}$
can be computed in expected time $\tilde{O}\left(m\log n\log\left(1/\nu\right)\right)$.
\end{lemma}
It should now become clear how we can combine a good chain with the
Algorithm \texttt{PreconditionedLogDetMonteCarlo}. We start by building
a chain. The partial Cholesky factorizations at each step of the chain
provide an upper bound to $\mbox{ld}\left(A\right)$. We then refine
this upper bound by running \texttt{PreconditionedLogDetMonteCarlo}
at each state of the chain to approximate $\mbox{ld}\left(B_{i}^{+}A_{i}\right)$
with high probability. The complete algorithm is presented in Algorithm
\ref{alg:The-main-algorithm}. We now have all the tools required
to prove Theorem \ref{thm:ultra_main}.

\textbf{Proof of Theorem \ref{thm:ultra_main}.} First, recall that
we can consider either an SDD or its grounded Laplacian thanks to
the relation $\log A=\ld L_{A}$. Call $A_{1}=L_{A}$ the first element
of the chain. In this proof, all the matrices will be Laplacian from
now on. Using Lemma \ref{lem:good-chain}, consider $\mathcal{C}=\left\{ A_{1}=A,B_{1},A_{2},\dots A_{d}\right\} $
a good chain for $A$, with $d=\mathcal{O}\left(\log n\right)$. More
precisely, since $A_{i+1}$= \texttt{GreedyElimination}$\left(B_{i}\right)$,
the Laplacian $B_{i}$ can be factored as: 
\[
B_{i}=P_{i}L_{i}\left(\begin{array}{cc}
D^{\left(i\right)} & 0\\
0 & A_{i+1}
\end{array}\right)L_{i}^{T}P_{i}^{T}
\]
with $P_{i}$ a permutation matrix, $L_{i}$ a lower triangular matrix
with $1$ one the diagonal and $D^{\left(i\right)}$a positive definite
diagonal matrix. The matrix $D^{\left(i\right)}$ is an immediate
by-product of running the algorithm \texttt{GreedyElimination} and
can be obtained when forming the chain $\mathcal{C}$ at no additional
cost.

From the discussion at the start of the section, it is clear that
$\ld B_{i}=\sum_{k}\log D_{k}^{\left(i\right)}+\ld A_{i+1}$. From
the discussion in Section \ref{sec:Preconditioned-log-determinants},
the log-determinant of $A$ is: 
\begin{eqnarray*}
\log A & = & \ld A_{1}\\
 & = & \ld B_{1}+\ld\left(B_{1}^{+}A_{1}\right)\\
 & = & \sum_{k}\log D_{k}^{\left(1\right)}+\ld A_{i+1}+\ld\left(B_{1}^{+}A_{1}\right)\\
 & = & \ld A_{d}+\sum_{i=1}^{d}\left(\sum_{k}\log D_{k}^{\left(i\right)}\right)+\sum_{i=1}^{d}\ld\left(B_{i}^{+}A_{i}\right)
\end{eqnarray*}
The term $\ld A_{d}$ can be estimated by dense Cholesky factorization
at cost $\mathcal{O}\left(1\right)$, and the diagonal Cholesky terms
$\sum_{k}\log D_{k}^{\left(i\right)}$ are already computed from the
chain. We are left with estimating the $d$ remainders $\text{ld}\left(B_{i}^{+}A_{i}\right)$.
By construction, $A_{i}\preceq B_{i}\preceq\kappa_{i}A_{i}$ and by
Lemma \ref{lem:linear-precond-existence}, there exists an operator
$C_{i}$ so that $\left\Vert C_{i}\left(b\right)-B_{i}^{+}b\right\Vert _{B_{i}}<\nu\left\Vert B_{i}^{+}b\right\Vert _{B_{i}}$
for all $b$ with a choice of relative precision $\nu=\frac{\epsilon}{16\kappa_{i}^{3}\kappa\left(B_{i}\right)}$.

This relative precision depends on the condition number $\kappa\left(B_{i}\right)$
of $B_{i}$. We can coarsely relate this condition number to the condition
number of $A_{1}$by noting the following: 
\begin{itemize}
\item Since $A_{i}\preceq B_{i}\preceq\kappa_{i}A_{i}$ by construction,
$\kappa\left(B_{i}\right)\leq\kappa_{i}\kappa\left(A_{i}\right)$ 
\item For diagonally dominant matrices or Laplacian matrices, the condition
number of the partial Cholesky factor is bounded by the condition
number of the original matrix. This can be seen by analyzing one update
in the Cholesky factorization. Given a partially factorized matrix
$\tilde{A}=\left(\begin{array}{ccc}
I_{p} & 0 & 0\\
0 & a & b\\
0 & b^{T} & S
\end{array}\right)$, after factorization, the next matrix is $\left(\begin{array}{cc}
I_{p+1} & 0\\
0 & S-a^{-1}bb^{T}
\end{array}\right)$. The spectrum of the Schur complement $S-a^{-1}bb^{T}$is bounded
by the spectrum of $\left(\begin{array}{cc}
a & b\\
b^{T} & S
\end{array}\right)$ (see Corollary 2.3 in \cite{Zhang2005}) and thus its condition number
is upper bounded by that of $\tilde{A}$. 
\end{itemize}
As a consequence, we have for all $i$: $\kappa\left(A_{i+1}\right)\leq\kappa\left(B_{i}\right)\leq\kappa_{i}\kappa\left(A_{i}\right)\leq\prod_{j=1}^{i}\kappa_{j}\kappa\left(A_{1}\right)=\tilde{O}\left(\kappa_{1}\kappa_{c}^{i-1}\kappa\left(A\right)\right)$
with $\kappa_{c}$ the constant introduced in Lemma \ref{lem:good-chain}.
This coarse analysis gives us the bound: 
\[
\kappa\left(B_{i}\right)\leq\tilde{O}\left(\kappa_{c}^{\log n}\log^{2}n~\kappa\left(A\right)\right)=\tilde{O}\left(n^{\log\kappa_{c}}\log^{2}n\ \kappa\left(A\right)\right).
\]
Consider the relative precision $\tilde{\nu}=\tilde{\mathcal{O}}\left(n^{-\log\kappa_{c}}\log^{-8}n\frac{\epsilon}{\kappa\left(A\right)}\right)$
so that $\tilde{\nu}\leq\nu_{i}$ for all $i$. Constructing the operator
$C_{i}$ is a byproduct of forming the chain \emph{$\mathcal{C}$.
}By Theorem \ref{thm:det-sampling-theorem}, each reminder $\text{ld}\left(B_{i}^{+}A_{i}\right)$
can be approximated to precision $\epsilon$ with probability at least
$1-\eta$ using Algorithm \ref{alg:SampleLogDet}. Furthermore, this
algorithm works in expected time 
\begin{eqnarray*}
 &  & \tilde{O}\left(m\log n\log\left(1/\tilde{\nu}\right)\kappa_{1}\left(\frac{1}{\epsilon}+\frac{1}{n\epsilon^{2}}\right)\log\left(\frac{n\kappa_{1}}{\tilde{\nu}}\right)\log^{2}\left(\kappa_{1}\right)\log\left(\eta^{-1}\right)\right)\\
 &  & =\tilde{O}\left(m\log^{3}n\left(\frac{1}{\epsilon}+\frac{1}{n\epsilon^{2}}\right)\log^{2}\left(\frac{n\kappa\left(A\right)}{\epsilon}\right)\log\left(\eta^{-1}\right)\right)
\end{eqnarray*}
By a union bound, the result also holds on the sum of all the $\log n$
approximations of the remainders. We can simplify this bound a little
by assuming that $\epsilon\geq n^{-1}$, which then becomes $\tilde{O}\left(m\epsilon^{-1}\log^{3}n\log^{2}\left(\frac{n\kappa\left(A\right)}{\epsilon}\right)\log\left(\eta^{-1}\right)\right)$. 
