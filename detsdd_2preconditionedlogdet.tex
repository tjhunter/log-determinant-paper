
\section{Preconditioned log-determinants\label{sec:Preconditioned-log-determinants}}

We begin by considering a simple sampling algorithm to compute log-determinants,
presented first in \cite{Barry1999}. We will first present some error
bounds on this algorithm that expand on bounds previously presented
in \cite{Bai1996} and \cite{Barry1999}.

Consider a real symmetric matrix $S\in\mathcal{S}_{n}^{+}$ such that
its spectral radius is less than $1$: $0\preceq S\preceq\left(1-\delta\right)I$
for some $\delta\in\left(0,1\right)$. We want to compute $\log\left|I-S\right|$
up to precision $\epsilon$ and with high probability. From the Martin
expansion: 
\begin{equation}
\log\left|I-S\right|=-\mbox{Tr}\left(\sum_{k=1}^{\infty}\frac{1}{k}S^{k}\right)\label{eq:martin}
\end{equation}


This series of traces can be estimated by Monte Carlo sampling, up
to precision $\epsilon$ with high probability. In order to bound
the errors, we will bound the large deviation errors using Hoeffding
inequality. We use a modified version of Proposition 4.2 from \cite{Bai1996}: 

\begin{lemma} \label{lem:hoeffding-trace}Consider $H\in\mathcal{S}_{n}$
lower- and upper-bounded by $\lambda_{\text{min}}$ and $\lambda_{\text{max}}$:
$\lambda_{\min}I\preceq H\preceq\lambda_{\max}I$. Consider $p$ vectors
sampled from the standard Normal distribution: $\mathbf{u}_{i}\sim\mathcal{N}\left(\mathbf{0},I_{n}\right)$
for $i=1\cdots p$. Then for all $\nu>0$: 
\[
\mathbb{P}\left[\left|\frac{1}{p}\sum_{i=1}^{p}\mathbf{u}_{i}^{T}H\mathbf{u}_{i}^{T}-\mbox{Tr}\left(H\right)\right|\geq\frac{\nu}{p}\right]\leq2\exp\left(-\frac{2\nu^{2}}{p\left(\lambda_{\max}-\lambda_{\min}\right)^{2}}\right)
\]
\end{lemma}

\begin{proof} This is a simple adaptation from \cite{Bai1996}. \end{proof}

We use the following Bernstein inequality:

\begin{lemma} \label{lem:bernstein}$X_{1}\cdots X_{U}$ are independant
random variables with $\mathbb{E}\left[X_{u}\right]=0$, $\left|X_{u}\right|<c$.
Call $\sigma^{2}=\frac{1}{U}\sum_{u}\text{Var}\left(X_{u}\right)$,
then for all $\epsilon>0$: 
\[
\mathbb{P}\left[\frac{1}{U}\sum_{u}X_{u}\geq\epsilon\right]\leq\exp\left(-\frac{U\epsilon^{2}}{2\sigma^{2}+2c\epsilon/3}\right)
\]


\end{lemma} The proof of this result can be found in a lecture note
from Peter Bartlett's class (stat241B/spring 2003). We can adapt some
results from \cite{Barry1999} to prove this bound on the deviation
from the trace. \begin{lemma} \label{lem:bernstein-trace}Consider
$H\in\mathcal{S}_{n}$, and call $\lambda_{i}$ the eigenvalues of
$H$. Consider $p$ vectors sampled from the standard Normal distribution:
$\mathbf{u}_{i}\sim\mathcal{N}\left(\mathbf{0},I_{n}\right)$ for
$i=1\cdots p$. Then for all $\epsilon>0$: 
\[
\mathbb{P}\left[\left|\frac{1}{p}\sum_{i=1}^{p}\frac{\mathbf{u}_{i}^{T}H\mathbf{u}_{i}}{\mathbf{u}_{i}^{T}\mathbf{u}_{i}}-\frac{1}{n}\mbox{Tr}\left(H\right)\right|\geq\epsilon\right]\leq\exp\left(-\frac{p\epsilon^{2}}{2n^{-1}\left(\lambda_{\max}-\lambda_{\min}\right)^{2}+\left(\lambda_{\max}-\lambda_{\min}\right)\epsilon}\right)
\]
with $\lambda_{\text{max}}=\max\lambda_{i}$ and $\lambda_{\min}=\min\lambda_{i}$.\end{lemma}
\begin{proof} The distribution of $\mathbf{u}_{i}$ is invariant
through a rotation, so we can consider $H$ diagonal. We assume without
loss of generality that $H=\text{diag}\left(\lambda_{1}...\lambda_{n}\right)$.
Again without loss of generality, we assume that $\lambda'_{\max}=\lambda_{\max}-\lambda_{\min}$
and $\lambda'_{\min}=0$ (by considering $H'=H-\lambda_{\min}I$).
Call $V_{i}=\frac{\mathbf{u}_{i}^{T}H\mathbf{u}_{i}}{\mathbf{u}_{i}^{T}\mathbf{u}_{i}}-n^{-1}\text{Tr}\left(H\right)$.
Using results from \cite{Barry1999}, we have: $\left|V_{i}\right|\leq\lambda_{\max}-\lambda_{\min}$,
$\mathbb{E}\left[V_{i}\right]=0$ and: 
\[
\text{Var}\left(V_{i}\right)=\frac{2}{n^{2}}\sum_{i=1}^{n}\left(\lambda_{i}-n^{-1}\text{Tr}\left(H\right)\right)^{2}
\]


Each of the variables $V_{i}$ is independent, so invoking Lemma \ref{lem:bernstein}
gives: 
\[
\mathbb{P}\left[\frac{1}{p}\sum_{i=1}^{p}V_{i}\geq\epsilon\right]\leq\exp\left(-\frac{p\epsilon^{2}}{2\sigma^{2}+2\left(\lambda_{\max}-\lambda_{\min}\right)\epsilon/3}\right)
\]
with $\sigma^{2}=\frac{2}{n^{2}}\sum_{i=1}^{n}\left(\lambda_{i}-n^{-1}\text{Tr}\left(H\right)\right)^{2}$
\end{proof} The previous lemma shows that if the eigenspectrum of
a matrix is bounded, we can obtain a Hoeffding bound on the error
done by sampling the trace. Furthermore, the convergence of the series
\ref{eq:martin} is also determned by the extremal eigenvalues of
$S$. If we truncate the series (\ref{eq:martin}), we can bound the
truncation error using the extremal eigenvalues. We formalize this
intuition in the following theorem, which is adapted from the main
Theorem in \cite{Barry1999}. While that main Theorem in \cite{Barry1999}
only considered a confidence interval based on the covariance properties
of Gaussian distribution, we generalize this result to more general
Hoeffding bounds. \begin{theorem} \label{thm:det-sampling-theorem}Consider
$S\in\mathcal{S}_{n}^{+}$ with $0\preceq S\preceq\left(1-\delta\right)I$
for some $\delta\in\left(0,1\right)$. Call $y$ the quantity to estimate:
\[
y=\log\left|I-S\right|
\]
and consider $\mathbf{u}_{i}\sim\mathcal{N}\left(\mathbf{0},I_{n}\right)$
for $i=1\cdots p$. Call $\hat{y}_{p,l}$ an estimator of the truncated
series of $l$ elements computed by sampling the trace using $p$
samples: 
\[
\hat{y}_{p,l}=\frac{n}{p}\sum_{j=1}^{p}\sum_{k=1}^{l}\frac{1}{k}\frac{\mathbf{u}_{j}^{T}S^{k}\mathbf{u}_{j}}{\mathbf{u}_{j}^{T}\mathbf{u}_{j}}
\]
By choosing $p\geq4\log\left(\eta^{-1}\right)\epsilon^{-2}\left(2n^{-1}\log^{2}\left(\delta^{-1}\right)+\epsilon\log\left(\delta^{-1}\right)\right)$
and $l\geq2\delta^{-1}\log\left(\frac{n}{\delta\epsilon}\right)$,
the error probability is small: 
\[
\mathbb{P}\left[\left|y-\hat{y}_{p,l}\right|\geq n\epsilon\right]\leq\eta
\]
\end{theorem} \begin{proof} The proof of this theorem follows the
proof of the Main Theorem in \cite{Barry1999} with some slight modifications.
Using triangular inequality: 
\[
\left|y-\hat{y}_{p,l}\right|\leq\left|\mathbb{E}\left[\hat{y}_{p,l}\right]-\hat{y}_{p,l}\right|+\left|y-\mathbb{E}\left[\hat{y}_{p,l}\right]\right|
\]


Since $S$ is upper-bounded, we have for all $k\in\mathbb{N}$: 
\[
\left|\mbox{Tr}\left(S^{k}\right)\right|\leq n\left(1-\delta\right)^{k}
\]
Using again triangle inequality, we can bound the error with respect
to the expected value: 
\begin{eqnarray*}
\left|y-\mathbb{E}\left[\hat{y}_{p,l}\right]\right| & = & \left|\sum_{i=l+1}^{\infty}\frac{\left(-1\right)^{i}}{i}\mbox{Tr}\left(S^{k}\right)\right|\\
 & \leq & \sum_{i=l+1}^{\infty}\frac{1}{i}\left|\mbox{Tr}\left(S^{k}\right)\right|\\
 & \leq & \frac{1}{l+1}\sum_{i=l+1}^{\infty}\left|\mbox{Tr}\left(S^{k}\right)\right|\\
 & \leq & \frac{n}{l+1}\sum_{i=l+1}^{\infty}\left(1-\delta\right)^{k}\\
 & \leq & \frac{n}{l+1}\frac{\left(1-\delta\right)^{l+1}}{\delta}\\
 & \leq & n\frac{\left(1-\delta\right)^{l+1}}{\delta}
\end{eqnarray*}


Hence, for a choice of $l\geq\delta^{-1}\log\left(\frac{2n}{\epsilon\delta}\right)$,
the latter part is less than $\epsilon/2$. We now bound the first
part using Lemma \ref{lem:bernstein-trace}. Call $H$ the truncated
series: 
\[
H=-\sum_{i=1}^{m}\frac{1}{i}S^{i}
\]
This truncated series is upper-bounded by $0$. The lowest eigenvalue
of the truncated series can be lower-bounded in terms of $\delta$:
\[
H=-\sum_{i=1}^{m}\frac{1}{i}S^{i}\succeq-\sum_{i=1}^{m}\frac{1}{i}\left(1-\delta\right)^{i}I\succeq-\sum_{i=1}^{+\infty}\frac{1}{i}\left(1-\delta\right)^{i}I\succeq-\sum_{i=1}^{+\infty}\frac{1}{i}\left(1-\delta\right)^{i}I=\left(\log\delta\right)I
\]
We can now invoke Lemma \ref{lem:bernstein-trace} to conclude: 
\[
\mathbb{P}\left[\left|\frac{1}{p}\sum_{i=1}^{p}\left(\mathbf{u}_{i}^{T}\mathbf{u}_{i}\right)^{-1}\mathbf{u}_{i}^{T}H\mathbf{u}_{i}-n^{-1}\mbox{Tr}\left(H\right)\right|\geq\frac{\epsilon}{2}\right]\leq\exp\left(-\frac{p\epsilon^{2}}{8n^{-1}\left(\log\delta\right)^{2}+4\log\left(1/\delta\right)\epsilon}\right)
\]
and $\exp\left(-\frac{p\epsilon^{2}}{8n^{-1}\left(\log\delta\right)^{2}+4\log\left(1/\delta\right)\epsilon}\right)\leq\eta$.
This latter inequality is verified for any 
\[
p\geq4\log\left(\eta^{-1}\right)\epsilon^{-2}\left(2n^{-1}\log^{2}\left(\delta^{-1}\right)+\epsilon\log\left(\delta^{-1}\right)\right)
\]


\end{proof} We immediately get the following result that justifies
the notion of preconditioners for determinants. The corresponding
algorithm, which we call \texttt{PreconditionedLogDetMonteCarlo},
is presented in Algorithm \ref{alg:SampleLogDet}. \begin{corollary}
\label{cor:preconditioning}If $A$ and $B$ are positive definite
matrices so that $B$ is a $\kappa-$approximation of $A$: 
\begin{equation}
A\preceq B\preceq\kappa A\label{eq:A-B-bounds}
\end{equation}
then the algorithm \texttt{PreconditionedLogDetMonteCarlo} computes
an $\epsilon$-approximation of $\frac{1}{n}\log\left|B^{-1}A\right|$
with high probability, by performing $O\left(\kappa\epsilon^{-2}\left(\log\kappa\right)^{2}\log\left(\frac{n^{2}\kappa}{\epsilon}\right)\right)$
vector inversions from $B$ and vector multiplies from $A$. \begin{proof}
We introduce some notations that will prove useful for the rest of
the article: 
\[
H=I-B^{-1}A
\]
\[
S=I-B^{-1/2}AB^{-1/2}
\]
with $B^{-1/2}$ the inverse of the square root of the positive-definite
matrix $B$%
\footnote{Given a real PSD matrix $X$, which can be diagonalized: $X=Q\Delta Q^{T}$
with $\Delta$ diagonal, and $\Delta_{ii}\geq0$. Call $Y=Q\sqrt{\Delta}Q^{T}$
the square root of $X$, then $Y^{2}=X$.%
}. The inequality \ref{eq:A-B-bounds} is equivalent to $\kappa^{-1}B\preceq A\preceq B$,
or also: 
\[
\left(1-\kappa^{-1}\right)I\succeq I-B^{-1/2}AB^{-1/2}\succeq0
\]
\[
\left(1-\kappa^{-1}\right)I\succeq S\succeq0
\]


\end{proof} \end{corollary} \begin{proof} The matrix $S$ is a
contraction, and its spectral radius is determined by $\kappa$. Furthermore,
computing the determinant of $B^{-1}A$ is equivalent to computing
the determinant of $I-S$: 
\begin{eqnarray*}
\log\left|I-S\right| & = & \log\left|B^{-1/2}AB^{-1/2}\right|\\
 & = & \log\left|A\right|-\log\left|B\right|\\
 & = & \log\left|B^{-1}A\right|\\
 & = & \log\left|I-H\right|
\end{eqnarray*}


and invoking Theorem \ref{thm:det-sampling-theorem} gives us bounds
on the number of calls to matrix-vector multiplies with respect to
$S$. It would seem at this point that computing the inverse square
root of $B$ is required, undermining our effort. However, we can
reorganize the terms in the series expansion to yield only full inverses
of $B$. Indeed, given $l\in\mathbb{N}^{*}$, consider the truncated
series: 
\begin{eqnarray*}
y_{l} & = & -\mbox{Tr}\left(\sum_{i=1}^{l}\frac{1}{i}S^{i}\right)\\
 & = & -\sum_{i=1}^{l}\frac{1}{i}\mbox{Tr}\left(S^{i}\right)\\
 & = & -\sum_{i=1}^{l}\frac{1}{i}\mbox{Tr}\left(\sum_{j}\left(\begin{array}{c}
j\\
i-j
\end{array}\right)\left(B^{-1/2}AB^{-1/2}\right)^{j}\right)\\
 & = & -\sum_{i=1}^{l}\frac{1}{i}\sum_{j}\left(\begin{array}{c}
j\\
i-j
\end{array}\right)\mbox{Tr}\left(\left(B^{-1/2}AB^{-1/2}\right)^{j}\right)\\
 & = & -\sum_{i=1}^{l}\frac{1}{i}\sum_{j}\left(\begin{array}{c}
j\\
i-j
\end{array}\right)\mbox{Tr}\left(\left(B^{-1}A\right)^{j}\right)\\
 & = & -\sum_{i=1}^{l}\frac{1}{i}\mbox{Tr}\left(\sum_{j}\left(\begin{array}{c}
j\\
i-j
\end{array}\right)\left(B^{-1}A\right)^{j}\right)\\
 & = & -\sum_{i=1}^{l}\frac{1}{i}\mbox{Tr}\left(H^{i}\right)
\end{eqnarray*}
Hence, the practical computation of the latter sum can be done on
$A^{-1}B$. To conclude, if we compute $p=4\epsilon^{-2}\left(\log\kappa\right)^{2}\log n$
truncated chains of length $l=\kappa\left(2\log n+\log\kappa+\log\left(\epsilon^{-1}\right)\right)$,
we get our result. This requires $lp$ multiplies by $A$ and inversions
by $B$. \end{proof} 

Usually, computing the exact inverse by an SDD matrix is too expensive.
We can instead extend the previous result to consider a black bock
that approximatively inverts compute $B^{-1}x$. If the error introduced
by the approximate inversion is limited, the result from the previous
corollary still holds. This is wha the following theorem establishes:

\begin{theorem}\label{thm:preconditioning-approx}Consider $A,B\in\mathcal{S}_{n}^{+*}$
with $B$ a $\kappa-$approximation of $A$. Furthermore, assume there
exists a function $C$ so that for all $y\in\mathbb{R}^{n}$, $C$
returns an $\epsilon-$approximation of $B^{-1}y$:
\[
\left\Vert BC\left(y\right)-y\right\Vert _{B}\leq\left\Vert B^{-1}y\right\Vert _{B}\epsilon
\]
Given $\eta>0$ and$\mu>0$, if $\epsilon\leq XXXX$, then the algorithm
\texttt{PreconditionedLogDetMonteCarlo} returns a scalar $z$ so that:
\[
\mathbb{P}\left[n^{-1}\left|z-\log\left|B^{-1}A\right|\right|\geq\mu\right]\leq\eta
\]
by performing $O\left(YYYY\right)$ vector calls to the operator $C$
and vector multiplies from $A$. 

\end{theorem}

The proof of this result is detailed in Appendix A.

This last theorem shows that we can compute a good approximation of
the log-determinant if the preconditioner $B$: (a) is close to $A$
in the spectral sense, and (b) can be approximately inverted and the
error introduced by the approximate inversion can be controlled. This
happens to be the case for symmetric, diagonally dominant matrices.
