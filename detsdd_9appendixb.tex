
\section*{Appendix B: Proofs of Section \ref{sec:Making-the-problem}}

We put here the proofs that pertain to Section \ref{sec:Making-the-problem}. 


\subsection{Properties of the generalized Laplacian}

Proof of Lemma \ref{lem:floating-properties}.

\begin{proof}The first statement is obvious from the construction
of the grounded Laplacian.

Statment (2) is a direct consequence of the fact that $F_{Z}=PZP^{T}$
with $P=\left(I_{n}\,0\right)$.

Then the third statement is a simple consequence of statement 2, as
$\text{ld}\left(Z\right)=\sum_{i}\log\lambda_{i}$ with $\left(\lambda_{i}\right)_{i}$
the $n-1$ positive eigenvalues of $Z$.

Statement (4) is straightforward after observing that the floating
procedure is a linear transform from $\mathcal{S}_{n}$ to $\mathcal{S}_{n-1}$,
so it preserves the matrix inequalities.

\end{proof}


\subsection{Technical lemmas for Theorem \ref{thm:ultra_main}}

This lemma generalizes Lemma 8.1 in \cite{Spielman2009a}.

\begin{lemma}\label{lem:simple-inequality}Consider $A\in\mathcal{S}_{n}$
positive semi-definite, and $x\in\mathbb{R}^{n}$. Then $xx^{T}\preceq\left(x^{T}A^{+}x\right)A$

\end{lemma}

\begin{proof}Without loss of generality, consider $x^{T}x=1$. Consider
the eigenvalue decomposition of $A$: $A=\sum_{i}\lambda_{i}u_{i}u_{i}^{T}$.
Since $\left(u_{i}\right)_{i}$ is an orthonormal basis of $\mathbb{R}^{n}$,
we only need to establish that $\left(u_{i}^{T}x\right)^{2}\leq\left(x^{T}A^{+}x\right)u_{i}^{T}Au_{i}$
for all $i$. The latter term can be simplified:
\begin{eqnarray*}
\left(x^{T}A^{+}x\right)u_{i}^{T}Au_{i} & = & \left(x^{T}\left[\sum_{j}\lambda_{j}^{-1}u_{j}u_{j}^{T}\right]x\right)\lambda_{i}\\
 & = & \lambda_{i}\sum_{j}\lambda_{j}^{-1}\left(u_{j}^{T}x\right)^{2}\\
 & \geq & \left(u_{i}^{T}x\right)^{2}
\end{eqnarray*}
which is the inequality we wanted.

\end{proof}

\begin{lemma} \label{lem:Jensen-inequality-matrix-logarithm-1}Jensen
inequality for the matrix logarithm. Let $A\in\mathcal{S}_{n}$ be
a positive semi-definite matrix with $p$ positive eigenvalues. Then
\[
\text{ld}\left(A\right)\leq p\log\left(\frac{\text{Tr}\left(A\right)}{p}\right)
\]
\end{lemma}

\begin{proof}This is a direct application of Jensen's inequality.
Call $\left(\lambda_{i}\right)_{i}$ the positive eigenvalues of $A$.
Then $\text{ld}\left(A\right)=\sum_{i}\log\lambda_{i}$. By concavity
of the logarithm:
\[
\sum_{i}\log\lambda_{i}\leq p\log\left(\frac{\sum\lambda_{i}}{p}\right)=p\log\left(\frac{\text{Tr}\left(A\right)}{p}\right)
\]


\end{proof}
