
\section{Structural bounds on the log-determinant}

How good is the estimate provided by the tree $T$? Intuitively, this
depends on how well the tree $T$ approximates the graph $G$. This
notion of quality of approximation can be formalized by the notion
of \emph{stretch}. \begin{definition} Stretch of a graph. Consider
$G$ a weighted graph, and $T$ a spanning tree of $G$. The stretch
of each edge $e$ from $G$ is defined as $\text{st}_{T}\left(e\right)=\omega_{e}\left(\sum_{f\in P}\frac{1}{\omega_{f}}\right)$
where $\omega_{e}$ is the weight of edge $e$ in $G$, and $P$ is
the unique path between the endpoints of $e$ in the tree $T$. The
stretch of the graph is the sum of the stretches of each edge: 
\[
\text{st}_{T}\left(G\right)=\sum_{e\in G}\text{st}_{T}\left(e\right)
\]


\end{definition} The stretch can be obtained as a by-product of the
computation of low-stretch spanning trees, of which the construction
can be done in $O\left(m\log n+n\log^{2}n\right)$ \cite{Abraham2008}.
Knowing the stretch gives upper and lower bounds on the value of the
log-determinant. \begin{proposition} Stretch bounds on the tree PLD:
\end{proposition} 
\begin{equation}
\log\left|T\right|+\left(n-1\right)\log\left(\frac{\text{st}_{G}\left(T\right)}{n-1}\right)\geq\log\left|G\right|\geq\log\left|T\right|+\log\left(\text{st}_{T}\left(G\right)-n+2\right)\label{eq:encadrement}
\end{equation}


\begin{proof} This is an application of Jensen's inequality on $\text{ld}\left(T^{+}G\right)$.
We have $\text{ld}G=\text{ld}T+\text{ld}\left(T^{+}G\right)$ and
$\text{ld}\left(T^{+}G\right)=\text{ld}\left(\sqrt{T}^{+}G\sqrt{T}^{+}\right)=\text{Tr}\left(\log^{+}\left(\sqrt{T}^{+}G\sqrt{T}^{+}\right)\right)$
with $\sqrt{T}$ the matrix square root of $T$. From Lemma \ref{lem:Jensen-inequality-matrix-logarithm},
we have 
\begin{eqnarray*}
\text{Tr}\left(\log^{+}\left(\sqrt{T}^{+}G\sqrt{T}^{+}\right)\right) & \leq & \left(n-1\right)\log\left(\frac{\text{Tr}\left(\sqrt{T}^{+}G\sqrt{T}^{+}\right)}{n-1}\right)\\
 & = & \left(n-1\right)\log\left(\frac{\text{Tr}\left(T^{+}G\right)}{n-1}\right)\\
 & = & \left(n-1\right)\log\left(\frac{\text{st}_{G}\left(T\right)}{n-1}\right)
\end{eqnarray*}


The lower bound is slightly more involved. Call $\lambda_{i}$ the
positive eigenvalues of $\sqrt{T^{+}}G\sqrt{T^{+}}$ and $\sigma=\text{st}_{T}\left(G\right)$.
We have $1\leq\lambda_{i}\leq\sigma$ because of the inequality$T\preceq G\preceq\text{st}_{T}\left(G\right)T$.
There are precisely $n-1$ positive eigenvalues $\lambda_{i}$, and
we have: $\text{ld}\left(T^{+}G\right)=\sum_{i}\log\lambda_{i}$.
One can show that $\sum_{i}\log\lambda_{i}\geq\log\left(\sigma-n+2\right)$
by considering the problem of minimizing $\sum_{i}\log x_{i}$ under
the constraints $\sum_{i}x_{i}=\sigma$ and $x_{i}\geq1$. This bound
is tight. \end{proof} Intuitively since the stretch of a tree is
bounded my $O\left(m\log n\right)$, it means that a low-stretch spanning
tree provides an approximation of a log-determinant up to a factor
of $O\left(n\log n\right)$. This could be of interest for nearly-cut
graphs, with large condition numbers.



\begin{lemma} \label{lem:Jensen-inequality-matrix-logarithm}Jensen
inequality for the matrix logarithm. Given $A\in\mathcal{S}_{n}^{+},$
$0\prec A\prec2$, the following inequalities hold: 
\[
\log\left(\frac{\text{Tr}\left(A\right)}{n}\right)\geq\frac{1}{n}\text{Tr}\left(\log A\right)
\]
\[
\text{ld}\left(\frac{\text{Tr}\left(A\right)}{n-1}\right)\geq\frac{1}{n-1}\text{Tr}\left(\log A\right)
\]
with $\log A=\sum_{k\geq1}\frac{1}{k}\left(I-A\right)^{k}$\end{lemma}
\begin{proof} Consider the diagonalization of $A$: $A=P\Delta P^{T}$
with $\Delta$ a diagonal (positive) matrix, and $P$ an orthogonal
matrix. Then 
\[
\log A=\sum_{k\geq1}\frac{1}{k}\left(PP^{T}-P\Delta P^{T}\right)^{k}=\sum_{k\geq1}\frac{1}{k}\left[P\left(I-\Delta\right)P^{T}\right]^{k}=P\left[\sum_{k\geq1}\frac{1}{k}\left(I-\Delta\right)^{k}\right]P^{T}=P\Gamma P^{T}
\]
with $\Gamma$ a diagonal matrix that verifies $\Gamma_{ii}=\log\Delta_{ii}$.
We can then conclude using the concavity of the logarithm over the
reals: 
\[
\log\left(\frac{\text{Tr}\left(A\right)}{n}\right)=\log\left(\frac{\sum_{i}\Delta_{ii}}{n}\right)\geq\frac{1}{n}\sum_{i}\log\Delta_{ii}=\frac{1}{n}\text{Tr}\left(\Gamma\right)=\frac{1}{n}\text{Tr}\left(\log A\right)
\]


The same reasoning holds for the pseudo-log-determinant while considering
all but one eigenvalue\end{proof} \begin{lemma} Power relation for
the matrix log: Given $A\in\mathcal{S}_{n}^{+},$ $0\prec A\prec2$,
and $k\in\mathbb{N}$, then $\log\left(A^{k}\right)=k\log A$ \end{lemma}


\subsection{Stretch of a graph\label{sub:Stretch-graph}}

We introduce here a refinement on an upper bound that is a byproduct
of using the algorithm XXX. \begin{definition} We define the \textbf{stretch
of a graph} $G=\left(V,E,\omega\right)$ with respect to a subgraph
$H=\left(V,\tilde{E},\tilde{\omega}\right)\subset G$ by the weighted
sum of effective resistances of edges $e\in E$ with respect to the
graph $H$: 
\[
\text{st}_{H}\left(G\right)=\sum_{e\in E}\omega_{e}\text{eff}_{H}\left(e\right)
\]


\end{definition} This is a generalization of the notion of stretch
defined by Alon, Karp, Peleg and West in .... We can use this definition
to generalize theorem 2.1 in \cite{Spielman2009b} \begin{theorem}
Let $G=\left(V,E,\omega\right)$ be a connected graph and let $H=\left(V,\tilde{E},\tilde{\omega}\right)$
be a connected subgraph of $G$. Let $L_{G}$ and $L_{H}$ be the
Laplacian matrices of $G$ and $H$ respectively. Then: 
\[
\text{Tr}\left(L_{H}^{+}L_{G}\right)=\text{st}_{H}\left(G\right)
\]
where $L_{H}^{+}$ is the pseudo inverse of $L_{H}$.\end{theorem}
\begin{proof} The proof is nearly identical to that of \cite{Spielman2009b},
except for the last line:

\begin{eqnarray*}
\text{Tr}\left(L_{H}^{+}L_{G}\right) & = & \sum_{\left(u,v\right)\in E}\omega\left(u,v\right)\text{Tr}\left(L_{\left(u,v\right)}L_{H}^{+}\right)\\
 & = & \sum_{\left(u,v\right)\in E}\omega\left(u,v\right)\text{Tr}\left(\left(\psi_{u}-\psi_{v}\right)\left(\psi_{u}-\psi_{v}\right)^{T}L_{H}^{+}\right)\\
 & = & \sum_{\left(u,v\right)\in E}\omega\left(u,v\right)\left(\psi_{u}-\psi_{v}\right)^{T}L_{H}^{+}\left(\psi_{u}-\psi_{v}\right)
\end{eqnarray*}
and the latter term is the effective resistance between $u$ and $v$
in the graph $H$: 
\begin{eqnarray*}
 & = & \sum_{\left(u,v\right)\in E}\omega\left(u,v\right)\text{eff}_{H}\left(u,v\right)\\
 & = & \text{st}_{H}\left(G\right)
\end{eqnarray*}


\end{proof} From a practical perspective, the graph stretch can be
computed in $\tilde{O}\left(m\log n/\epsilon^{2}\right)$ \begin{proposition}
There exists an algorithm that computes an $\epsilon$-approximation
of $\text{st}_{H}\left(G\right)$ in $\tilde{O}\left(m\log n/\epsilon^{2}\right)$\end{proposition}
\begin{proof} This is the second main result from Spielman and Srivastana
(XXX cite). \end{proof}
